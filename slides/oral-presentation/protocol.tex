\section{Protocol Detail}

\subsection{Flowchart}
\begin{frame}{Flowchart}
	\begin{center}
		\resizebox{!}{.6\textwidth}
		{%
			\begin{tikzpicture}[node distance=2cm]
				% Define flow charts' component
				\node (init) [process] {Initial};
				\node (lock) [process, below of=init] {Lock};
				\node (twostep) [process, below of=lock] {Request \& Response};
				\node (twostep_left) [process, left of=twostep, xshift=-2cm] {Request \& Response};
				\node (twostep_right) [process, right of=twostep, xshift=2cm] {Request \& Response};
				\node (voting) [decision, below of=twostep] {Voting};
				\node (store) [process, below of=voting, yshift=-0.5cm] {Store ACK};
				\node (update) [io, below of=store, align=center] {File \\ Transmit};
				\node (auditing) [process, right of=update, xshift=2.5cm] {Auditing};
				\node (unlock) [process, left of=store, xshift=-4cm] {Unlock};

				% Define flow charts' link
				\path [line](init) -- (lock);
				\path [line](lock) -- (twostep);
				\path [line](lock) -- (twostep_left);
				\path [line](lock) -- (twostep_right);
				\path [line](twostep) -- (voting);
				\path [line](twostep_left) -- (voting);
				\path [line](twostep_right) -- (voting);
				\path [line](voting) -- node[anchor=east] {k個以上的}
									    node[anchor=west] {ACK相同} (store);
				\path [line](voting) -| node[anchor=west] {ACK不相同} (auditing);
				\path [line](store) -- (update);
				\path [line](store) -- (unlock);
				\path [line](unlock) |- (lock);
			\end{tikzpicture}%
		}
	\end{center}
\end{frame}

\subsection{Download \& Upload}
\begin{frame}{Download \& Upload}
	\centering
	\textcolor{blue}{\textbf{Request \& Response}}\\
	~\\
	~\\
	\begin{minipage}{.45\textwidth}
        \includegraphics[width=1\textwidth]{2_step_handshake}
    \end{minipage}%
	\begin{minipage}{.55\textwidth}
    	\footnotesize
		\centering
		\begin{equation*} \label{eq1}
                \begin{split}
                        \textcolor{blue}{REQ}\ & =\ (OP,\ SN,\ [OP,\ SN]_{pri(D)}) \\
                        OP\ & =\ (TYPE,\ PATH,\ HASH) \\
                        SN\ & =\ Sequence\ Number
                \end{split}
        \end{equation*}
        \divider{}\\
        \begin{equation*} \label{eq2}
                \begin{split}
                        \textcolor{red}{ACK}\ & =\ (RESULT,\ REQ,\\
                        & \tab{}\tab{}[RESULT,\ REQ]_{pri(S)}) \\
                        RESULT\ & =\ (roothash,\ filehash)
                \end{split}
        \end{equation*}
        \textcolor{red}{\textbf{collect ACKs and voting}}
    \end{minipage}%
\end{frame}

\begin{frame}{if Operation is UPLOAD}
	\centering
	\textcolor{blue}{\textbf{Server Update Merkle tree}}\\
	~\\
	~\\
	\begin{center}
		\includegraphics[width=.65\textwidth]{update_merkle_tree}
	\end{center}
\end{frame}

\begin{frame}{File Transmit}
	\begin{center}
		\includegraphics[width=.65\textwidth]{file_transmit}
	\end{center}
\end{frame}

\subsection{Audit}
\begin{frame}{Audit}
	\textcolor{blue}{$\because$ ACK 中有 roothash\\
	$\therefore$ 新的 request 之前,所有的檔案都經過檢查,沒有問題}\\
	~\\	
	\begin{block}{}
		\centering
		device request $OP_i$,收到回傳的 $ACK_i$\\
		發現 $Server_p$ 的 ACK 有錯誤,因此向 $Server_p$ 稽核\\
		~\\
		device 向 $Server_p$ 索取 $MT_{i-1}$\\
		($MT_{i-1}$ 為執行 $OP_i$ 之前的 Merkle tree)
	\end{block}
	\begin{alertblock}{\encircle{1}\encircle{2} 兩點有一個出錯就能確定 $Server_p$ 出錯}
		\encircle{1} device 檢查 $MT_{i-1}$ 的 roothash 應和 $ACK_{i-1}$ 中的 roothash 一樣\\
		\encircle{2} device 以 $OP_i$ 中的 hash value 來更新 $MT_{i-1}$,\\
		\tab{}更新後的 roothash 應和 $Server_p$ 現在的 roothash 相同
	\end{alertblock}	
\end{frame}