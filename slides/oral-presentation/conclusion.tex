\section{Conclusion and Future Work}

\begin{frame}{Conclusion}
	我們提出了一個新的方法,使雲端儲存系統能夠及時地發動稽核以及POV。我們考量一個使用者帳號的檔案可能由多個客戶端設備同時操作存取等等動作的情況。客戶端的設備不需要暫存任何的檔案雜湊值或是檔案狀態的資訊。
    ~\\
    ~\\
	我們利用多個獨立的服務提供者,使用者每一次的操作都向所有服務提供者發送請求指令,收集所有服務提供者的回傳資料後經比對能夠及時的確認資料的完整性,而回傳資料上的簽章及密碼學的證據能夠達到 POV 的效果,當發生問題時使用者和服務提供者雙方能夠以保留的證據稽核確認發生錯誤的是哪一方。
    ~\\
    ~\\
    實驗結果顯示,本論文提出的方法相較於之前的雲端儲存即時稽核方法,平均來看能夠節省7倍以上的時間,遇到最糟的情況能夠節省高達將近50倍的時間。而且本論文提出的方法解決了之前方法會遭遇的一種最壞的情況,即使用者在長時間未上線,下一次上線時要同步非常多動作所花的時間。
\end{frame}

\begin{frame}{Future Work}
	1. 我們希望能夠將 FBH 樹套用到本論文的方法中,藉由實驗觀察能否增快Merkle tree在更新檔案時的速度。
    ~\\
    ~\\
    2. 在本論文中同步伺服器的功能是確保各個設備之間動作的順序性,若有新的演算法能不需依賴同步伺服器,將能使我們的架構更彈性且使用更少的硬體。
\end{frame}

\begin{frame}
	\begin{center}
		\includegraphics[width=\textwidth]{thank_you.jpg}
	\end{center}	
\end{frame}