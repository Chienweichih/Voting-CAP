\section{Conclusion and Future Work}

\begin{frame}{Conclusion}
	我們提出了一個應用於雲端儲存的 Real-time POV 技術,利用投票的方式快速檢查 Data Integrity,也即時的將資料備份到多個 Server 上。\\
    ~\\
    ~\\
    實驗結果顯示,相較於之前的 Real-time POV 技術,平均能夠節省 8 倍的時間,Worst-case 時更能夠節省超過 20 倍的時間。\\
    ~\\
    ~\\
    雲端儲存系統可以使用本論文提出的方法,提供雙方不可否認的保證於他們的服務層級協議(Service Level Agreement)中。
\end{frame}

\begin{frame}{Future Work}
	\begin{enumerate}
		\item 我們希望能將 FBHTree\footnotemark 套用到本論文的方法中,藉由實驗觀察能否增快 Merkle tree 在更新檔案時的速度。\\
        ~\\
        \item 在本論文中使用同步伺服器來維護 Write Serializability,若有新的演算法能夠不需依賴同步伺服器又能維護 Write Serializability,將能讓我們的架構更加彈性且使用更少的硬體。
	\end{enumerate}
    \footnotetext{G.-H. Hwang and H.-F. Chen, "Efficient Real-time Auditing and Proof of Violation for Cloud Storage Systems," in 9th IEEE International Conference on Cloud Computing, San Francisco, USA, 2016.}
\end{frame}

\begin{frame}{Thanks for Your Listening}
	\begin{center}
		\includegraphics[width=\textwidth]{thank_you.jpg}
	\end{center}	
\end{frame}